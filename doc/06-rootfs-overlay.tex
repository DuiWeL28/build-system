\section{The rootfs overlay}
\paragraph \\
If the blend is the brains of heads, then the rootfs overlay is the heart of
heads\footnotemark . This overlay is a git repository residing in the root
directory of heads' filesystem. It holds all specific configuration files,
scripts, sources, and other miscellaneous files that are a part of heads.
\url{https://git.devuan.org/heads/rootfs-overlay}

\subsection{Scripts}
\paragraph \\
heads-specific scripts usually reside in \emph{/usr/local/bin/heads-*}.
They are mostly small helper scripts that I will document here for the
sake of reference.

\subsubsection{heads-generate-passphrase}
\paragraph \\
Generates a random passphrase for the root user on every boot. Drops it once
in luther's .zshrc and shows it when one logs in. It is never shown again
and is lost unless it's remembered or changed. Executed by \emph{/etc/rc.local}.

\subsubsection{heads-init}
\paragraph \\
A tiny case clause, used by shutdown-menu to either shut down or reboot.

\subsubsection{heads-shutdown-menu}
\paragraph \\
A script that utilizes zenity\footnotemark to give a graphical shutdown
menu from AwesomeWM.

\subsubsection{heads-torify}
\paragraph \\
The iptables black magic we use to route all network traffic through Tor.
Executed by \emph{/etc/rc.local}.

\subsubsection{heads-update}
\paragraph \\
A script to update the system. Explained in the \emph{updating heads} section.


\subsection{Updating heads}
\paragraph \\
As mentioned, the rootfs overlay is a git repository. We can utilize this to
provide seamless updates to users, by pushing commits to the rootfs overlay git
repository. Keep in mind heads as such is \textbf{never} going to phone
home\footnotemark , but it's up to the user to choose if they will update or
not. In any case, these will mostly be minor updates from heads' side, and if
there are any important (read: security) updates, we will rather roll out a new
release than relying on people updating their rootfs overlay.

\paragraph \\
However, since git is awesome (read: decentralized), it's easy to host
one's own rootfs overlay if one wants to customize their heads setup, have a
certain kind of persistence in the system or just easily grab files when
one boots into the live system. The possibilites are endless.


\footnotetext[9]{such organs, much anatomy}
\footnotetext[10]{\url{https://help.gnome.org/users/zenity/stable/}}
\footnotetext[11]{\url{https://en.wikipedia.org/wiki/Phoning_home}}
