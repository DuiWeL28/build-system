\section{Building heads}
\paragraph \\
A little bit of spoonfeeding is always nice. Let's go through the process
of building a heads ISO... Learning this process is essential, but don't
worry, it's the easiest process in the world of building ISOs.

\subsection{Obtaining the build-system}

\paragraph \\
I use a Devuan based system to bootstrap heads images, and I can't promise
it's going to work anywhere else (It should, but I haven't tried it ever,
nor do I plan to). So let's dive into this beautiful process of obtaining
the heads build system and baking our ISO :)

\paragraph \\
First off, we need to grab the \emph{build-system} git repository. It
contains git submodules, so we append --recursive to the args:
\begin{verbatim}
$ git clone https://git.devuan.org/heads/build-system.git --recursive
\end{verbatim}

This will give us all we need. Check out the README to find out what
dependencies you need to install to run it on your system. If you are
on the master git branch, you are using the development version - that's
what you should use for testing and development, but if you require an
image for daily usage, you should probably checkout the latest tag you
can find in the git history.

\subsection{Firing up live-sdk}
\paragraph \\
Okay, so once we have our git repo cloned and possibly checked out, we
go into the \emph{live-sdk} directory of the repo. From there you start
a vanilla zsh shell, source live-sdk to enter its environment and initialize
the blend to build heads:
\begin{verbatim}
$ cd live-sdk
$ zsh -f
$ source sdk
$ load devuan amd64 heads
\end{verbatim}

Once done, you will be notified all is loaded and ready to run. There is a
helper command that will run the following functions in sequence, and exit
if one of them exits. It is called \textbf{build\_iso\_dist}.

\subsubsection{bootstrap\_complete\_base}
\paragraph \\
This function is the one that provides us with a vanilla base system that
is as minimal as it can be, and then we build up upon it. It uses our
\emph{extra\_packages} array and installs our packages. The function also
creates a tarball of the minimal system so it is not needed to debootstrap
for every build. If the tarball is found when starting a build, it will be
extracted and updated.

\subsubsection{blend\_preinst}
\paragraph \\
This one was mentioned before. It is blend-specific, and currently only adds
the "luther" user.

\subsubsection{iso\_prepare\_strap}
\paragraph \\
A function that will install the packages needed for a liveCD to function
properly using \emph{apt}.

\subsubsection{build\_kernel\_arch}
\paragraph \\
Mentioned before as well. This will clone (or update) the latest
\emph{linux-heads} sources, compile them, and install them into the heads
bootstrap directory (our filesystem). The function exists in libdevuansdk,
but is overriden with the heads blend.

\subsubsection{iso\_setup\_isolinux}
\paragraph \\
After the kernel is installed, we have to copy it for isolinux to use it.
This function also copies the needed isolinux binaries where they have to
be.

\subsubsection{iso\_write\_isolinux\_cfg}
\paragraph \\
This function drops the isolinux configuration file into the isolinux directory.
Exists in libdevuansdk, but is overriden with the blend because of branding.

\subsubsection{blend\_postinst}
\paragraph \\
Mentioned in an earlier chapter. This function calls other
\emph{blend\_install\_software} functions to compile and install specific
software like Tor, GNU IceCat and such from source, because their packages
aren't provided in Devuan. It also clones the \emph{rootfs-overlay} and sets
it up. Finally, it executes the hacks and fixes that can be found in the
\emph{blend\_finalize} function, and cleans up the filesystem, making it ready
for packing. The function is blend-specific.

\subsubsection{iso\_squash\_strap}
\paragraph \\
After our filesystem is ready, this will create a SquashFS from the bootstrap
directory. This is what holds our read-only live system.

\subsubsection{iso\_xorriso\_build}
\paragraph \\
Finally, this function creates a bootable ISO from our SquashFS and the ISOLINUX
files we have copied earlier. The ISO can be found in \emph{dist/} after xorriso
is finished.
